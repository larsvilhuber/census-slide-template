% $Id: Presentation-appendix.tex 1704 2015-07-28 01:46:49Z lv39 $ 
% $URL: https://forge.cornell.edu/svn/repos/lv39_papers/SynLBD/text/JSM2015-final/Presentation/Presentation-appendix.tex $ 

\section{Extra slides}

\begin{frame}
\begin{block}{Extra slides}

\end{block}
\end{frame}


\begin{frame}
\frametitle{Bibliography}
\tiny
\bibliography{../abbrev,../paper}
\bibliographystyle{IEEEtranS}
\end{frame}

\subsection*{Acronyms}
\begin{frame}{Acronyms}

\input{../acronyms}

\end{frame}

\subsection{SynLBD Details}
\label{sec:SynLBD_details}

\section[Access]{Access to SynLBD}
\begin{frame}{Feedback loop}
\begin{block}{Critical element}
\begin{itemize}[<+->]
\item Not just ``release and forget''
\item First attempt, needs feedback 
\item Researchers want reassurance
\end{itemize}
\end{block}
\begin{block}{Closing the loop}
\begin{itemize}
\item Researchers access the data on a special server (open internet, no RDC)
\item No disclosure-avoidance analysis done on results created from SynLBD
\item Validation server allows to request validation, release of results using confidential data (offline submission, full disclosure-avoidance)
\end{itemize}
\end{block}
\end{frame}

\begin{frame}{Access to SynLBD}
\begin{block}{Key goals}
\begin{itemize}
\item Easier (very easy) access for researchers: average project approval within 2 (TWO) week
\item Quick turnaround on validation (depends on complexity)
\item See also SIPP Synthetic Beta (SSB)
\end{itemize}
\end{block}
\end{frame}


\begin{frame}{Application}
\begin{block}{Process to gain access}
\begin{itemize}
\item Abstract of a project
\item Description of the variables needed 
\item Application decisions  based solely on feasibility
\end{itemize}
\end{block}
\end{frame}




\begin{frame}{Validation}
\begin{block}{Validation is easy}
if the analysis runs error-free on the SDS, then researchers can request that programs be run against the confidential data. All such analyses are reviewed by Census Bureau Disclosure Review Officers, and approved output is provided to both the researchers as well as to the Statistics of Income (SOI) Program at the United States Internal Revenue Service (IRS).
\end{block}
\end{frame}


